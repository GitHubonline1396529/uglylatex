\documentclass[10pt, twocolumn]{uglypaper}
\title{\bfseries UglyPaper:一个“丑陋”的\LaTeX{}论文模板}
\author{GitHubonline1396529}
\date{\zhdate{2022/12/31}}

% 本文档命令
\addbibresource{reference.bib} % 参考文献,不要删除

\begin{document}

\maketitlewithonecolabstract{
    本文是UglyPaper模板的排版效果示例及模板文档,在展示排版效果的同时简要阐述了模板的部分功能及其使用方法。UglyPaper最初是我留作自用的\LaTeX 模板,具有较好的Pandoc Markdown to PDF via \LaTeX 兼容性,适合用于日常写作的快速排版。

    \keywords{\LaTeX;\quad 排版;\quad 文档类;}
}

\section{UglyPaper使用须知}

\subsection{UglyPaper模板介绍}

自从\href{https://github.com/ElegantLaTeX/}{Elegant\LaTeX}项目停更之后,我就时常感到十分的无措,因为我原本很喜欢这个项目,系列模板使用起来也特别方便,尤其是可以通过在Markdwon文件的YAML Header中使用\texttt{documentclass}指定文档类,再通过Pandoc一次性转换为PDF via \LaTeX{}快速排版。

最初,为了满足我个人的使用需求,我自己搓了这几个模板。后来觉得比较好用,我就觉得应该发出来跟大家分享,大家一起用。但是因为我的技术比较菜,而且没有什么艺术细胞,做不到Elegant,所以我把项目命名为了Ugly\LaTeX{},很合理吧。

\subsection{守正创新}

本模板延用了\href{https://github.com/ElegantLaTeX/}{Elegant\LaTeX}的部分功能的实现。尽管\href{https://github.com/ElegantLaTeX/}{Elegant\LaTeX}的部分功能 (比如多样化的颜色主题) 还没有实现出来,但是后续会逐渐增加。目前最基本最关键的已经有了。包括
\begin{itemize}
  \item \textbf{语言模式切换}:支持通过文档类选项\texttt{lang=cn}和\texttt{lang=en}切换中英文语言模式。
  \item \textbf{定理与公式环境}:支持数学公式编辑,并提供了11种不同的定理类环境的选项,支持交叉引用。
  \item \textbf{适配不同设备},包括 Pad,Screen (幻灯片),Kindle,PC (双页),通用 (A4 纸张);
\end{itemize}

除此之外,本项目还在\href{https://github.com/ElegantLaTeX/}{Elegant\LaTeX}的基础之上增加了一系列新的优势性功能,包括但不限于
\begin{itemize}
  \item \textbf{新增两种排版}:小开本 (32开A5),课本 (B5纸张);
  \item \textbf{更严谨更现代化的目录结构}:模块化功能便于维护,支持使用\texttt{Makefile}安装到目录。
  \item \textbf{Pandoc兼容性}:从Markdown文件快速构建您的文档PDF。
\end{itemize}

\subsection{语言模式}

本模板内含两套语言环境,改变语言环境会改变图表标题的引导词 (图,表),文章结构词 (比如目录,参考文献等),以及定理环境中的引导词 (比如定理,引理等)。不同语言模式的启用如下:

\begin{lstlisting}[
  frame=none, language=TeX,
  caption={启用语言模式}
  ]  
  \documentclass[cn]{elegantnote} 
  \documentclass[lang=cn]{elegantnote} 
  \documentclass[en]{elegantnote} 
  \documentclass[lang=en]{elegantnote}
\end{lstlisting}

\begin{note}
只有中文模式才可输入中文,如果需要在英文模式下输入中文,可以自行添加 \lstinline{ctex} 宏包\footnote{需要使用 \lstinline{scheme=plain} 选项才不会把标题改为中文。}或者使用 \lstinline{xeCJK} 宏包设置字体。另外如果在笔记中使用了抄录环境 (\lstinline{lstlisting}),并且里面有中文字符,请务必使用 \hologo{XeLaTeX} 编译。
\end{note}

\subsection{文档类选项}

此模板基于\LaTeX{}的标准文类article设计,所以article文类的选项也能传递给本模板,比如 \texttt{a4paper, 10pt} 等等。

\subsection{参考文献}

文献部分,本模板调用了biblatex宏包,并使用了biber,采用国标 GB7714-2015。

关于文献条目 (bib item),你可以在谷歌学术,Mendeley,Endnote 中取,然后把它们添加到 \texttt{reference.bib} 中。在文中引用的时候,引用它们的键值 (bib key) 即可。参考文献示例:\cite{cn1,en2,en3}使用了中国一个大型的 P2P 平台 (人人贷) 的数据来检验男性投资者和女性投资者在投资表现上是否有显著差异。

\subsection{定理类环境}

此模板采用了 \lstinline{amsthm} 中的定理样式,使用了 4 类定理样式,所包含的环境分别为
\begin{itemize}
  \item \textbf{定理类}:theorem,lemma,proposition,corollary;
  \item \textbf{定义类}:definition,conjecture,example;
  \item \textbf{备注类}:remark,note,case;
  \item \textbf{证明类}:proof。
\end{itemize}

\begin{remark}
在选用 \lstinline{lang=cn} 时,定理类环境的引导词全部会改为中文。
\end{remark}

\section{写作示例}

我们将通过三个步骤定义可测函数的积分。首先定义非负简单函数的积分。以下设 $E$ 是 $\mathcal{R}^n$ 中的可测集。

\begin{definition}[可积性]
设 $ f(x)=\sum\limits_{i=1}^{k} a_i \chi_{A_i}(x)$ 是 $E$ 上的非负简单函数,其中 $\{A_1,A_2,\ldots$, $A_k\}$ 是 $E$ 上的一个可测分割,$a_1,a_2,\ldots,a_k$ 是非负实数。定义 $f$ 在 $E$ 上的积分为 1. 3
\begin{equation}
   \label{inter}
   \int_{E} f dx = \sum_{i=1}^k a_i m(A_i).
\end{equation}
一般情况下 $0 \leq \int_{E} f dx \leq \infty$。若 $\int_{E} f dx < \infty$,则称 $f$ 在 $E$ 上可积。
\end{definition}

一个自然的问题是,Lebesgue 积分与我们所熟悉的 Riemann 积分有什么联系和区别?之后我们将详细讨论 Riemann 积分与 Lebesgue 积分的关系。这里只看一个简单的例子。设 $D(x)$ 是区间 $[0,1]$ 上的 Dirichlet 函数。即 $D(x)=\chi_{Q_0}(x)$,其中 $Q_0$ 表示 $[0,1]$ 中的有理数的全体。根据非负简单函数积分的定义,$D(x)$ 在 $[0,1]$ 上的 Lebesgue 积分为
\begin{equation}\label{inter2}
  \int_0^1 D(x)dx = \int_0^1 \chi_{Q_0} (x) dx = m(Q_0) = 0
\end{equation}
即 $D(x)$ 在 $[0,1]$ 上是 Lebesgue 可积的并且积分值为零。但 $D(x)$ 在 $[0,1]$ 上不是 Riemann 可积的。

\begin{table}[htbp]
  \centering
  \small
  \caption{燃油效率与汽车价格}
    \begin{tabular}{lcc}
    \toprule
                  &       (1)         &        (2)      \\
    \midrule
    燃油效率      &   -238.90***      &      -49.51     \\
                  &    (53.08)        &      (86.16)    \\
    汽车重量      &                   &        1.75***  \\
                  &                   &       (0.641)   \\
    常数项        &  11253.00***      &    1946.00      \\
                  &  (1171.00)        &   (3597.00)     \\
    观测数        &     74            &      74         \\
    $R^2$         &      0.220        &       0.293     \\
    \bottomrule
    \end{tabular}%
  \label{tab:reg}%
\end{table}%

\begin{theorem}[Fubini 定理]\label{thm:fubi}
若 $f(x,y)$ 是 $\mathcal{R}^p\times\mathcal{R}^q$ 上的非负可测函数,则对几乎处处的 $x\in \mathcal{R}^p$,$f(x,y)$ 作为 $y$ 的函数是 $\mathcal{R}^q$ 上的非负可测函数,$g(x)=\int_{\mathcal{R}^q}f(x,y) dy$ 是 $\mathcal{R}^p$ 上的非负可测函数。并且
\begin{equation}\label{eq:461}
  \int_{\mathcal{R}^p\times\mathcal{R}^q} f(x,y) dxdy=\int_{\mathcal{R}^p}\left(\int_{\mathcal{R}^q}f(x,y)dy\right)dx.
\end{equation}
\end{theorem}

\begin{proof}
Let $z$ be some element of $xH \cap yH$.  Then $z = xa$ for some $a \in H$, and $z = yb$ for some $b \in H$. If $h$ is any element of $H$ then $ah \in H$ and $a^{-1}h \in H$, since $H$ is a subgroup of $G$. But $zh = x(ah)$ and $xh = z(a^{-1}h)$ for all $h \in H$. Therefore $zH \subset xH$ and $xH \subset zH$, and thus $xH = zH$.  Similarly $yH = zH$, and thus $xH = yH$, as required.
\end{proof}


回归分析 (regression analysis) 是确定两种或两种以上变量间相互依赖的定量关系的一种统计分析方法。根据定理~\ref{thm:fubi},其运用十分广泛,回归分析按照涉及的变量的多少,分为一元回归和多元回归分析;按照因变量的多少,可分为简单回归分析和多重回归分析;按照自变量和因变量之间的关系类型,可分为线性回归分析和非线性回归分析。

% \nocite{*}
\printbibliography[
  % heading=bibintoc, 
  title=\ebibname]

\appendix
%\appendixpage
\addappheadtotoc

\end{document}
