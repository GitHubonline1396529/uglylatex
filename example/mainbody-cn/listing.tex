对于\texttt{listing}宏包的代码有两种展示方式:直接在文档中插入代码块,或者从代码文件输入。前者适用于较短的代码文件的展示,后者则更适合长代码文件的展示 (例如放在附录中长代码)。\texttt{listing}宏包的高亮模式已经在文档类中设置好。一个简单的\texttt{listling}代码块如下所示,使用\texttt{language}参数可以指定语言,使用\texttt{caption}参数可以设置标题。

% tex-fmt: off
\begin{minted}{tex}
\begin{lstlisting}[
  numbers=left, % 启用行号
  language=C, 
  caption={
    一个简单的Hello World程序
    \label{lst:insert}
    }
  ]
#include <stdio.h>

int main( void ) {
    /* Print 'hello, world' message. */
    print("hello, world\n");
    return 0;
}
}
\end{lstlisting}
\end{minted}
% tex-fmt: on

\begin{note}
  如果想要在文章的任意位置引用这段代码,可以像上面这样为标题增加一个\texttt{label}。但是要注意,\texttt{\textbackslash label\{\}}要紧跟在\texttt{caption}文本的后面。
\end{note}

上述代码的实际演示效果如代码块\ref{lst:insert}所示。

% \begin{figure*}[!h]
\begin{lstlisting}[
  numbers=left, % 启用行号
  language=C, 
  caption={
    一个简单的Hello World程序
    \label{lst:insert}
    }
  ]
#include <stdio.h>

int main( void ) {
    /* Print 'hello, world' message. */
    print("hello, world\n");
    return 0;
}
}
\end{lstlisting}
% \end{figure*}

\begin{note}
  如果你想要在双列模式下插入代码块,你可以尝试将代码块置于一个\texttt{figure*}环境内,这将会创建一个横跨两列的代码块。
\end{note}

% 以文件形式插入代码块的示例见附录代码\ref{lst:demo}。
