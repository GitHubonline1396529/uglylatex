本模板定义了如下的额外的功能性命令:

\begin{enumerate}
    \item \texttt{\textbackslash fig\{<图题>\}\{<图片文件名>\}\{<label内容>\}}:用于快速方便地插入图片。
    \item \texttt{\textbackslash subtitle\{\}}:用于为文档添加一个副标题是。
    \item \texttt{\textbackslash maketitlewithonecolabstract\{<摘要内容>\}}:在使用双列模式的情况下,同时实现插入摘要文本和生成文档标题。
    \item \texttt{\textbackslash shserifbold}:使用思源宋体粗体,通常用在文档的大标题上——如果你不喜欢黑体作为文档总标题的话。
\end{enumerate}

\begin{remark}
  使用\texttt{twocolumn} 参数的情况下,文档的大标题会被默认安置在文档的一侧。如果您希望大标题横跨两栏,就必须使用\texttt{\textbackslash twocolumn}命令的参数栏将\texttt{\textbackslash maketitle}命令和\texttt{onecolabstract}环境一起包裹起来:

% tex-fmt: off
\begin{minted}{tex}
\twocolumn[
  \maketitle
  \begin{onecolabstract}
    摘要内容……摘要内容……摘要内容……
    \end{onecolabstract}
]
\end{minted}
% tex-fmt: on

  这将会十分不方便。因此,为了解决这个问题,我们定义了专门的
% tex-fmt: off
\begin{minted}{tex}
maketitlewithonecolabstract\{<摘要内容>\}
\end{minted}
% tex-fmt: on
  命令,来实现two column abstract一步处理。
\end{remark}