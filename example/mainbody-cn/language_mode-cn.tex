本模板内含两套语言环境,改变语言环境会改变图表标题的引导词 (图,表),文章结构词 (比如目录,参考文献等),以及定理环境中的引导词 (比如定理,引理等)。不同语言模式的启用如下:

\begin{minted}{tex} 
  \documentclass[cn]{elegantnote} 
  \documentclass[lang=cn]{elegantnote} 
  \documentclass[en]{elegantnote} 
  \documentclass[lang=en]{elegantnote}
\end{minted}

\begin{note}
  % 只有中文模式才可输入中文,如果需要在英文模式下输入中文,可以自行添加 \lstinline{ctex} 宏包\footnote{需要使用 \lstinline{scheme=plain} 选项才不会把标题改为中文。}或者使用 \lstinline{xeCJK} 宏包设置字体。另外如果在笔记中使用了抄录环境 (\lstinline{lstlisting}),并且里面有中文字符,请务必使用 \hologo{XeLaTeX} 编译。
  无论是中文模式还是英文模式,都可以正常输入中文文本,而且全都使用\hologo{XeLaTeX}编译。
\end{note}