本文档类为充分兼容Markdwon to \LaTeX via PDF,为Markdown中的Callout Blocks/Adminition/Alart Quote Blocks\footnote{这些都是同一种Markdown扩展语法的别名}扩展提供了额外的格式支持。在文档类中,定义了\texttt{info}、\texttt{warning}、\texttt{tip}、\texttt{important}、\texttt{caution}五种Callout环境,以便兼容Markdown语法。

\begin{note}
  在GitHub上,使用\texttt{note}而不是\texttt{info}作为Callout块的名称,但是因为\texttt{note}在本文档类中被分配给了“注”环境,这里只能使用\texttt{info}。
\end{note}

具体来说,在一些扩展的Markdown语法当中,你通常可以使用如下的语句来指定一个带有醒目颜色标记的“警示块”\footnote{对于这一语法的详细使用方法,可以参考\href{}{GitHub}上的\href{https://github.com/orgs/community/discussions/16925}{这个 discussions 页面}。}。

\begin{minted}{markdown}
> [!warning]
> 这是一个信息块,你可以将一些信息写入文本当中,
> 这些信息就会被展示在框里。
\end{minted}

这一语法当前在GitHub、Obsidian、Typora、Visual Studio Code等平台上均可以使用,这就导致了大量的Markdown文件都包含这种语法。但到目前为止,Pandoc尚未提供这一语法的支持,如果直接将相应的Markdown文本转化为\LaTeX{},就会发现这些文本被“原样”地包含在导出的PDF当中\footnote{关于这种情况的详细信息,可以参考我的博客\href{https://www.cnblogs.com/BOXonline1396529/articles/18799607}{利用Pandoc Lua Filter将Markdown中的Alarts Quote Block渲染为\LaTeX{}环境}}。为了应对这种情况,本文档类的作者为用户提供了这个 Pandoc 的 Lua 过滤器脚本\href{https://github.com/GitHubonline1396529/callout2latex}{\texttt{callout2latex}},以便用户更加方便地将Markdown文件通过Pandoc转化为PDF via \LaTeX{}。

\begin{info}
  突出显示用户应该考虑的信息。
\end{info}

\begin{warning}
  由于存在潜在风险,需要用户立即关注的关键内容。
\end{warning}

\begin{tip}
  帮助用户取得更大成功的可选信息。
\end{tip}

\begin{important}
  某一行为的潜在负面后果。
\end{important}

\begin{caution}
  这是一条需要注意的内容。
\end{caution}

\begin{remark}
  尽管这个文档环境在本系列下属的三个文档类当中都是可以使用的,因为三个文档类文件共享同一份环境定义,但是我还是建议你仅在UglyNote文档类中使用,因为UglyPaper和UglyRep看上去更加正式,如果出现这些彩色的块,看上去就会很奇怪。
\end{remark}